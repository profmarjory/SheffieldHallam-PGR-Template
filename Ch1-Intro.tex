\chapter{Introduction}

\begin{figure}[!htb]
    \caption{Example of a figure}
    \centering 
    \includegraphics[scale=0.5]{img.png}
    \label{figure:img}
\end{figure}


Reference example \cite{Abreu:2010}.

\begin{markdown}

## A Markdown Heading

- **bold text**
- *italic text*
- [A link](https://shu.ac.uk/)

\end{markdown}

\begin{labelledTable}{LabelledTable}{|c|c|c|}
 \hline
 cell1 & cell2 & cell3 \\ 
 cell4 & cell5 & cell6 \\ 
 cell7 & cell8 & cell9 \\ 
 \hline
\end{labelledTable}

\begin{minted}[
frame=leftline,
framesep=2mm,
baselinestretch=1.1,
bgcolor=LightGray,
fontsize=\footnotesize,
linenos
]{python}
import numpy as np
    
def incmatrix(genl1,genl2):
    m = len(genl1)
    n = len(genl2)
    M = None #to become the incidence matrix
    VT = np.zeros((n*m,1), int)  #dummy variable
    
    #compute the bitwise xor matrix
    M1 = bitxormatrix(genl1)
    M2 = np.triu(bitxormatrix(genl2),1) 

    for i in range(m-1):
        for j in range(i+1, m):
            [r,c] = np.where(M2 == M1[i,j])
            for k in range(len(r)):
                VT[(i)*n + r[k]] = 1;
                VT[(i)*n + c[k]] = 1;
                VT[(j)*n + r[k]] = 1;
                VT[(j)*n + c[k]] = 1;
                
                if M is None:
                    M = np.copy(VT)
                else:
                    M = np.concatenate((M, VT), 1)
                
                VT = np.zeros((n*m,1), int)
    
    return M
\end{minted}
